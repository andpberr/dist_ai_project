\documentclass{beamer}
\usepackage{graphicx}
\usepackage{color}
%\graphicspath{{}}

\title{Using Abstractions to Solve Opportunistic Crime Security Games at Scale}
\author{C. Zhang, V. Bucarey, A. Mukhopadhyay, A. Sinha, Y. Qian, Y. Vorobeychik, M. Tambe}
\begin{document}
	\frame{\titlepage}
	\begin{frame}
		\frametitle{Outline}
		\begin{itemize}
			\item Related Work and Background
				\begin{itemize}
					\item Stackelberg and Opportunistic Security Games
					\item Machine Learning in Criminology
					\item Dynamic Bayesian Network
					\item Abstract Games
				\end{itemize}
			\item Contribution 1: Game Abstraction Framework
			\item Contribution 2: Layer-Generating Algorithm
			\item Contribution 3: Planning Algorithm
			\item Contribution 4: Heuristic Propagation Model
			\item Experimentation
				\begin{itemize}
					\item University of Southern California
					\item Nashville Metro Area
				\end{itemize}
			\item Summary
		\end{itemize}
	\end{frame}

	\begin{frame}
		\frametitle{Stackelberg Security Games}
		\begin{itemize}
			\item Stackelberg Game: Game in which a "leader" agent makes an initial move and the "follower" agent moves afterward
			\item Stackelberg Security Game: Patrolling officers as leader and criminal as follower
		\end{itemize}
	\end{frame}

	\begin{frame}
		\frametitle{Opportunistic Security Games}
		\begin{itemize}
			\item Model for opportunistic adversaries who:
			\begin{itemize}
				\item Opportunistically and repeatedly seek targets
				\item React to new information at execution rather than planning in advance
				\item Have limited observation of defender strategies
			\end{itemize}
		\end{itemize}
	\end{frame}

	\begin{frame}
		\frametitle{Security Games Pros and Cons}
		\begin{itemize}
			\item Pros
			\begin{itemize}
				\item Allow extensions to handle real world scenarios
			\end{itemize}
			\item Cons
			\begin{itemize}
				\item Adversary models built from expert input (lacking detail)
			\end{itemize}
		\end{itemize}
	\end{frame}

	\begin{frame}
		\frametitle{Machine Learning in Criminology}
		\begin{itemize}
			\item Analyze crime data and suggest police patrol strategies
			\item Difference: This paper also examines interplay between attacker/defender
		\end{itemize}
	\end{frame}

	\begin{frame}
		\frametitle{Dynamic Bayesian Network}
		A Bayesian Network which relates variables to each other from one time step to the next.\\
		\includegraphics[width=3cm, height=5cm]{dbn}
	\end{frame}

	\begin{frame}
		\frametitle{Dynamic Bayesian Network}
		\begin{itemize}
			\item Pros:
			\begin{itemize}
				\item Predicts criminal behavior with high accuracy
			\end{itemize}
			\item Cons:
			\begin{itemize}
				\item Cannot scale to problems with a large number of targets (i.e. most real world examples)
				\item Poor performance where defender's patrol data is limited
				\item Restricted only to pure strategies
			\end{itemize}

		\end{itemize}
	\end{frame}


	\begin{frame}
		\frametitle{Abstract Games}
		\begin{itemize}
			\item Abstract games in prior research focused on clustering actions, strategies or states
			\item This paper merges targets, propagates abstractions to those new targets, and learns from real-world data
		\end{itemize}
	\end{frame}

	\begin{frame}
		\frametitle{Notation}
		\begin{itemize}
			\item $ N $ - Number of targets \\
			\item $ T $ - Number of shifts \\
			\item $ n $ - Largest scale of problem that can be handled by learning and planning
			\item $ c^{i}_{1} $ - Number of crimes per shift at target $i$ with a defender present \\
			\item $ c^{i}_{0} $ - Number of crimes per shift at target $i$ without a defender present \\
			\item $ Dis_{ij} $ - Dissimilarity between $i$ and $j$ ($Dis_{ij} = |c^{i}_{1} - c^{j}_{1}| + |c^{i}_{0} - c^{j}_{0}|$) \\
			\item $ I $ - Set of targets in original layer \\
			\item $ K $ - size of partition of $I$ \\ 
			\item $ I_{k} $ - k-sized subset of $I$ (functions as the aggregated target in the abstract layer) \\
			\item $ Dis(I_{k}) $ - Inner dissimilarity of an aggregated target ($Dis(I_{k}) = \Sigma_{i,j \in I_{k}} Dis_{ij}$) \\
		\end{itemize}
	\end{frame}

	\begin{frame}
		\frametitle{Notation}
		\begin{itemize}
			\item $ d_{ij} $ - Physical distance between targets i and j \\
			\item $ In(I_{k}) $ - Inertia of an aggregated target ($In(I_{k}) = min_{j}\Sigma_{i\in I_{k}}d_{ij}$) \\
			\item $ \alpha $ - Normalization parameter \\
			\item $ L_{I}(K) $ - information loss function ($min_{\{I_{k}\}^{K}_{k=1}\in \mathcal{P}_{K}(I)}\Sigma^{K}_{k=1} \alpha In(I_{k}) + Dis(I_{k}) $) \\
			\item $ x_{j} $ - binary var; 1 if $j$ is the center of an aggregated target, 0 otherwise \\
			\item $ y_{ij} $ - 1 if target $i$ is in $j$-centered aggregated target, 0 otherwise \\
			\item $ z_{ik} $ - Equal to $Dis_{ik}$ if $i$ and $j$ are in the same aggregated target, 0 otherwise \\
			\item $ A $ - movement matrix \\
			\item $ B $ - crime matrix \\
			\item $ Att_{n} $ - attractiveness of target $n$
		\end{itemize}
	\end{frame}

	\begin{frame}
		\frametitle{Notation}
		\begin{itemize}
			\item $ \lambda $ - influence of num of criminals on criminal movement \\
			\item $ \mu $ - influence of num of defenders on criminal movement \\
		\end{itemize}
	\end{frame}

	\begin{frame}
		\frametitle{Contribution 1: Game Abstraction Framework}
		\begin{itemize}
			\item Previous research deals with opportunistic crime, but fails to scale
			\item This paper uses abstraction to address this
			\item Abstract layer generated from original 
				\begin{itemize}
					\item Contains aggregated targets (aggregated from similar targets in original)
				\end{itemize}
		\end{itemize}
		\includegraphics[width=3cm, height=5cm]{abstraction}
	\end{frame}


	\begin{frame}
		\frametitle{Contribution 1: Game Abstraction Framework}
		\begin{itemize}
			\item Multiple layers of abstraction are possible, if warranted
			\item Learning
			\item Strategizing (abstract)
			\item Propagation
			\item Strategy for Original Layer
		\end{itemize}
	\end{frame}

	\begin{frame}
		\frametitle{Contribution 2: Layer-Generating Algorithm}
		\begin{itemize}
			\item Districting problem - Division into ``balanced'' subregions
			\item Geometric Contstraints:
				\begin{itemize}
					\item Contiguity
					\item Compactness
					\item Environmental Constraints
				\end{itemize}
			\item Minimize dissimilarity within aggregated targets ($Dis_{ij}$)
			\item Consideration of Scalability Constraint

		\end{itemize}
	\end{frame}

	\begin{frame}
		\frametitle{Contribution 2: Layer-Generating Algorithm}
		\includegraphics[width=11cm, height=7cm]{generation_milp}
	\end{frame}

	\begin{frame}
		\frametitle{Contribution 2: Layer-Generating Algorithm}
		\includegraphics[width=11cm, height=7cm]{generation_algorithm}
	\end{frame}

	\begin{frame}
		\frametitle{Dynamic Bayesian Network}
		\begin{itemize}
			\item Defenders (D)
			\item Criminals (X)
			\item Crimes committed (Y)
			\item Nothing new; novelty comes from the propagation of adversary behavior parameters
		\end{itemize}
		\includegraphics[width=3cm, height=5cm]{dbn}
	\end{frame}

	\begin{frame}
		\frametitle{Contribution 3: Planning Algorithm}
		\begin{itemize}
			\item Focused on planning with mixed strategies
			\item No loss of optimality
			\item Less room for criminal exploitation
		\end{itemize}
		\includegraphics[width=10cm, height=6cm]{planning_optim}
	\end{frame}

	\begin{frame}
		\frametitle{Contribution 4: Heuristic Propagation Model}
		\begin{itemize}
			\item Direct learning (with sufficient data)
			\item Parameter propagation (when data is limited)
				\begin{itemize}
					\item Uses Quantal Response model
					\item Models probability of transition from one target to another as a function of attractiveness
					\item Using with DBN adds influence of defenders and criminals at previous time step
				\end{itemize}
		\end{itemize}
		\includegraphics[width=10cm, height=3cm]{propagation_optim}
	\end{frame}

	\begin{frame}
		\frametitle{Strategy Computation in Original Layer}
		\begin{itemize}
			\item Each aggregated target has its own DBN
			\item Parameters $A$ and $B$ are used when there is sufficient data
			\item $\lambda$, $\mu$, and $Att$ are used to reconstruct A and B, which are then used as DBN params
		\end{itemize}
	\end{frame}

	\begin{frame}
		\frametitle{Extension}
		\begin{itemize}
			\item When $n^2 < N$, there are too many targets for a 2-layer approach to work
			\item Nesting occurs, and we end up with $M$ layers
			\item $ M = \lfloor log_{n}N \rfloor + 1 $
			\item Learn and plan at layer $M$, propagate to $M-1$, learn and plan at layer $M-1$, ... propagate to layer $1$
		\end{itemize}
	\end{frame}

	\begin{frame}
		\frametitle{University of Southern California}
		\begin{itemize}
			\item $n=5$
			\item $N=25$
			\item Total crimes on the order of $10^2$
			\item 2 layer abstraction
			\item Limited patrol data (must learn $\lambda, \mu, Att$ and estimate parameters $A$ and $B$ )
		\end{itemize}
		\includegraphics[width=6cm, height=3cm]{usc}
	\end{frame}

	\begin{frame}
		\frametitle{Nashville Metro Area}
		\begin{itemize}
			\item $n=5$
			\item $N=900$
			\item Total crimes on the order of $10^3$
			\item $M=\lfloor log_{5}900 \rfloor + 1 = 5$ layer abstraction
			\item Detailed patrol data (GPS dispatches)
		\end{itemize}
		\includegraphics[width=3cm, height=3cm]{davidson}
	\end{frame}

	\begin{frame}
		\frametitle{Results}
		\includegraphics[width=10cm, height=6cm]{results}
	\end{frame}

	\begin{frame}
		\frametitle{Summary}
		\begin{itemize}
			\item Aggregate targets in original layer to produce abstract layer
			\item Conduct learning and develop patrol strategy in abstract layer
			\item Propagate parameters for use in the DBNs of the original layer
			\item Produce patrol strategy for original layer (mixed or pure)
		\end{itemize}
	\end{frame}

	\begin{frame}
		\frametitle{Questions?}
		%{\huge {\color{blue} Questions?}}
	\end{frame}

\end{document}
